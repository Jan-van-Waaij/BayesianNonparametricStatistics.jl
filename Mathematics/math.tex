\documentclass[12pt]{article}
\usepackage{jansstylefile} % The option nobibliography does not print the bibliography. But bibliography can still be used. 

\begin{document}

\section{Distribution of the posterior of a finite basis expansion with Gaussian coefficients}

\begin{lemma}
	Let \(X^T=\set{X_t:t\in[0,T]}\) be an observation of 
	\begin{align*}
		dX_t=b(X_t)dt+\sigma(X_t)dW_t,
	\end{align*}
	where \(b\) is equipped with the prior distribution defined by 
	\[
	b=\sum_{j=1}^k\theta_j\phi_j,
	\]
	where \(\set{\phi_1,\ldots,\phi_k}\) is a linearly independent basis, and \(\theta=(\theta_1,\ldots,\theta_k)^t\) has multivariate normal distribution \(N(\mu,\Sigma)\) and \(\sigma\) is a positive measurable function. Then the  posterior distribution of \(\theta\) is \(N(\hat\mu,\hat\Sigma)\), where \[\hat\mu=(S+\Sigma^{-1})^{-1}(m+\Sigma^{-1}\mu),\quad\hat\Sigma= (S+\Sigma^{-1})^{-1}\] and the vector \(m=(m_1,\ldots,m_k)^t\) is defined by 
	\[
	m_l=\int_0^T\frac{\phi_l(X_t)}{\sigma(X_t)^2}dX_t, \quad l=1,\ldots,k,
	\] 
	and the symmetric \(k\times k\)-matrix \(S\) is given by 
	\[
	S_{l,l'}=\int_0^T\frac{\phi_l(X_t)\phi_{l'}(X_t)}{\sigma^2(X_t)}dt,\quad l,l'=1,\ldots,k,
	\]
	provided \(S+\Sigma^{-1}\) is invertible. 
	Moreover the marginal likelihood is given by 
	\[
\int p(X^T\mid \theta)p(\theta)d\theta=	|\Sigma\hat\Sigma^{-1}|^{-1/2}e^{-\frac12\mu^t\Sigma^{-1}\mu} e^{\frac12\hat\mu^t\hat\Sigma^{-1}\hat\mu}
	\]
	
\end{lemma}
\begin{proof}Almost surely we have by Girsanov's theorem\begin{equation}\label{eq:girsanov}
p(X^T\mid \theta)=\exp\left(\int_0^T\frac{b(X_t)}{\sigma(X_t)^2}dX_t-\frac12\int_0^T\rh{\frac{b(X_t)}{\sigma(X_t)}}^2dt\right),
\end{equation}with respect to the Wiener measure. So \(\log p(X^T\mid b)=\theta^tm - \frac 1 2 \theta^t S\theta\). And the log of the distribution of \(\theta\) with respect to the Lebesgue measure on \(\re^k\) is given by
\begin{align*}
	\log p(\theta)= &-\frac k2\log(2\pi) - \frac12 \log|\Sigma|  - \frac 1 2 (\theta-\mu)^t\Sigma^{-1}(\theta-\mu) \\
	= &C_1 - \frac 1 2 \theta\Sigma^{-1}\theta +\theta^t\Sigma^{-1}\mu,
	\intertext{with}
	C_1=& -\frac k2\log(2\pi) - \frac12 \log|\Sigma|  - \frac 1 2 \mu^t\Sigma^{-1}\mu. 
\end{align*}


So, %by the Bayes formula, for some constant \(C_3\), the posterior density of \(\theta\) is given by
\begin{align*}
	\log[ p(X^T\mid \theta)p(\theta)] = & C_1 + \theta^tm - \frac 1 2 \theta^t S\theta - \frac 1 2 \theta\Sigma^{-1}\theta +\theta^t\Sigma^{-1}\mu\\
	= & C_1 + \theta^t ( m + \Sigma^{-1} \mu ) - \frac 1 2 \theta^t (S+\Sigma^{-1}) \theta\\
	= & C_1 + \theta^t ( S + \Sigma^{-1} )  \Big  ( ( S + \Sigma^{-1} )^{-1} (m + \Sigma^{-1}\mu )\Big) \\
	&\quad- \frac 1 2 \theta^t (S+\Sigma^{-1}) \theta. 
\end{align*}

By the Bayes formula, the posterior density of \(\theta\) is proportional to \(p(X^T\mid\theta)p(\theta)\). It follows that  \(\theta\mid X^T\) is normally distributed with mean 
\[\hat\mu :=( S + \Sigma^{-1} )^{-1} (m + \Sigma^{-1}\mu).\]
 and covariance matrix \[\hat\Sigma:=(S+\Sigma^{-1})^{-1}.\]
 
 Moreover 
 \begin{align*}
 &\int  p(X^T\mid\theta)p(\theta)d\theta \\
 = & \int  e^{C_1} e^{\theta^t\hat\Sigma^{-1}\hat\mu } e^{-\frac12\theta^t \hat\Sigma^{-1} \theta} d\theta\\
 = & (2\pi)^{k/2}| \hat\Sigma|^{1/2}e^{\frac12\hat\mu^t\hat\Sigma^{-1}\hat\mu}e^{C_1}\\
 &\times \int (2\pi)^{-k/2}|\hat\Sigma|^{-1/2}e^{\theta^t\hat\Sigma^{-1}\hat\mu } e^{-\frac12\theta^t \hat\Sigma^{-1} \theta} e^{-\frac12\hat\mu^t\hat\Sigma^{-1}\hat\mu}d\theta\\
 = & (2\pi)^{k/2}|\hat\Sigma|^{1/2}e^{\frac12\hat\mu^t\hat\Sigma^{-1}\hat\mu}e^{C_1}\\
 = & |\Sigma\hat\Sigma^{-1}|^{-1/2}e^{-\frac12\mu^t\Sigma^{-1}\mu} e^{\frac12\hat\mu^t\hat\Sigma^{-1}\hat\mu}.
 \end{align*}
 using that the integrant is a probability distribution.
\end{proof}

\section{The marginal maximum likelihood estimator}

Suppose we have prior \(\theta\sim N(0,\Sigma_\lambda)\), where \(\Sigma_\lambda=\lambda^2\Sigma \). Note that 
  \[
\Sigma_\lambda \hat \Sigma_\lambda ^{-1} = \Sigma_\lambda (S+\Sigma_\lambda ^{-1})= \Sigma_\lambda  S + \II_k=\lambda^2\Sigma  S + \II_k
\]
and\[
\hat \mu^t \hat\Sigma_\lambda^{-1} \hat\mu = m^t (S+\Sigma_\lambda^{-1})^{-1}m= m^t (S+\lambda^{-2}\Sigma^{-1})^{-1}m. 
\]

So 
\begin{align*}
&\log \int  p(X^T\mid\theta)p(\theta)d\theta \\
= & \log|\lambda^2 \Sigma S + \II_k| + \frac12 m^t (S+\lambda^{-2}\Sigma^{-1})^{-1}m. 
\end{align*}





\begin{lemma}
	Let \(X^T=\set{X_t:t\in[0,T]}\) be an observation of 
	\begin{align*}
	dX_t=b(X_t)dt+\sigma(X_t)dW_t,
	\end{align*}
	where \(b\) is equipped with the prior distribution defined by 
	\begin{align*}
	\lambda^2 \sim & \text{Inverse Gamma}(A,B)=IG(A,B)\\
	\theta \mid \lambda \sim & N(\mu,\lambda^2\Sigma)\\
	b\mid \theta = & \sum_{j=1}^k\theta_j\phi_j,
	\end{align*}
	where \(\set{\phi_1,\ldots,\phi_k}\) is a linearly independent basis. Then 
	\[
	\lambda^2 \mid \theta, X^T \sim \text{IG}\rh{ A + k/2 , B+ \frac12 (\theta-\mu)^t\Sigma^{-1}(\theta-\mu) }.
	\] 
\end{lemma}
\begin{proof}
	Let where the vector \(m=(m_1,\ldots,m_k)^t\) is defined by 
	\[
	m_l=\int_0^T\frac{\phi_l(X_t)}{\sigma(X_t)^2}dX_t, \quad l=1,\ldots,k,
	\] 
	and the symmetric \(k\times k\)-matrix \(S\) is given by 
	\[
	S_{l,l'}=\int_0^T\frac{\phi_l(X_t)\phi_{l'}(X_t)}{\sigma^2(X_t)}dt,\quad l,l'=1,\ldots,k.
	\]
	Almost surely we have by Girsanov's theorem\begin{equation}\label{eq:girsanov}
	p(X^T\mid b)=\exp\left(\int_0^T\frac{b(X_t)}{\sigma(X_t)^2}dX_t-\frac12\int_0^T\rh{\frac{b(X_t)}{\sigma(X_t)}}^2dt\right),
	\end{equation}with respect to the Wiener measure. So \(\log p(X^T\mid b)=\theta^tm - \frac 1 2 \theta^t S\theta\). And the logarithm of the distribution of \(\theta\) given \(\lambda\) with respect to the Lebesgue measure on \(\re^k\) is given by (proportionality w.r.t. \(\lambda\)),
	\begin{align*}
	\log p(\theta\mid \lambda)= &C_1 -k\log \lambda - \frac 1 2 \lambda^{-2}(\theta-\mu)^t\Sigma^{-1}(\theta-\mu). 
	\end{align*}
	for some real constant \(C_1\), depending on \(\theta\), but not on \(\lambda\).
	
	In the following, \(\propto\) means equal up to a multiplicative constant depending on \(\theta\) and \(X^T\), but not on \(\lambda\).
	By the Bayes formula, \begin{align*}
		p(\lambda^2\mid \theta, X^T)\propto & p(X^T\mid \lambda^2,\theta)p(\lambda^2\mid \theta)
		\intertext{and}
		p(\lambda^2\mid \theta)\propto & p(\theta\mid \lambda^2)p(\lambda^2)\intertext{so}
		p(\lambda^2\mid \theta, X^T)\propto &  p(X^T\mid \lambda^2,\theta)p(\theta\mid \lambda^2)p(\lambda^2).
	\end{align*}It follows that for some real constants \(C,\tilde C\) depending on \(\theta\) and \(X^T\), but not on \(\lambda\), we have \begin{align*}
		&\log p(\lambda^2\mid \theta, X^T)\\ 
		= & C + \theta^tm - \frac 1 2\theta^t S\theta\\
		& -k\log \lambda - \frac12 \lambda^{-2}(\theta-\mu)^t\Sigma^{-1}(\theta-\mu)\\
		&-(A+1)\log(\lambda^2) - \frac B{\lambda^2}\\
		= &\tilde{C} -(A+k/2+1)\log(\lambda^2) - \frac {B+ \frac12 (\theta-\mu)^t\Sigma^{-1}(\theta-\mu)}{\lambda^2}.
	\end{align*}
	Which is the logarithm of the density of the inverse gamma distribution with shape parameter \(A+k/2\) and scale parameter \(B+ \frac12 (\theta-\mu)^t\Sigma^{-1}(\theta-\mu)\).
\end{proof}

\begin{lemma}
	We have \[
	p(X^T\mid j,\lambda^2)= \frac{\expa{\frac12 m^T \rh{S+(\lambda^2 \Sigma)^{-1}}^{-1}m}}{\sqrt{\det\rh{\lambda^2\rh{S+(\lambda^2 \Sigma)^{-1}}\Sigma}}}.
	\]
\end{lemma}
\begin{proof}
	This follows from \[
	p(X^T\mid j,\lambda^2 ) = \int p(X^T\mid j,\theta^j,\lambda^2)p(\theta^j\mid j, \lambda)d\theta^j.
	\]
\end{proof}
\section{Number of dependent Faber-Schauder functions with higher or equal index}

Note that for level \(j\ge 1\), \(\psi_{j,k}\) and \(\psi_{j,l}\) are only dependent when \(k=l\) (obviously, then they are equal). 

Note that \(\psi_1\) and \(\psi_{0,1}\) are dependent, both of level 0. 

When \(d\ge 1\), then there are \(2^d\) Faber functions of level \(j+d\) that are dependent with \(\psi_{j,k}\), \(j\ge 0\). These are 
\[
\psi_{j+d,(k-1)2^d+1},\psi_{j+d,(k-1)2^d+2},\ldots,\psi_{j+d,k2^d}
\]

Every Faber-Schauder function is obviously dependent with itself. 

Indexing with \(i=2^j+k\), when \(\psi_{j,k}\) has index \((j,k)\) (excluding \(i=1\)), we see that, when \(j\ge 0\), \(\psi_i\) is dependent with \(2^{j'-j}\), functions \(\psi_{j',k'}\), \(i'=2^{j'}+k'\ge i\) of level \(j'\ge j\)  (including itself, when \(j'=j\)).

So if \(J\) is the higest level, \(\psi_i\) is dependent with 
\[
\sum_{d=0}^{J-j}2^d=2^{J-j+1}-1.
\]
Faber-Schauder functions \(\psi_{i'}\) with index \(i'\ge i\). 
Hence summing over all levels \(0,\ldots,J\) and indices within a level, the number of combinations of functions \((\psi_{j,k},\psi_{j',k'}),0\le j,j'\le J\) and  \(i=2^j+k\le 2^{j'}+k'=i'\) which are dependent is
\begin{align*}
\sum_{j=0}^J\sum_{k=1}^{2^j}(2^{J-j+1}-1)\\
=\sum_{j=0}^J(2^{J+1}-2^j)\\
=(J+1)2^{J+1}-(2^{J+1}-1)\\
=J2^{J+1}+1. 
\end{align*}

The Faber-Schauder function \(\psi_1\) is dependent with every Faber-Schauder function (including itself) up to and including level \(J\), which counts for \(2^{J+1}\) Faber-Schauder functions with a higher index or equal index, up to level \(J\). 

In total we have 

\[
J2^{J+1}+1+2^{J+1}=(J+1)2^{J+1}+1. 
\]
Faber-Schauder functions up to level \(J\) dependent with a Faber-Schauder function with equal (itself) or higher index. 


If we only consider dependent pairs \((\psi_i,\psi_{i'})\) with \(i'>i\), then we have 
\[
J2^{J+1}+1
\]
of such pairs (minus all \(2^{J+1}\) diagonal pairs \((\psi_i,\psi_i)\)).

Hence, by symmetry, there are in total \(J2^{J+1}+1+J2^{J+1}+1+2^{J+1}=(2J+1)2^{J+1}+2\) pairs \((\psi_i,\psi_{i'})\) that are dependent.

\begin{lemma}
	The Girsanov covariantie matrix is sparse.
\end{lemma}
\begin{proof}
	At most \((2J+1)2^{J+1}+2\) entries of the \(2^{J+1}\times 2^{J+1}\)-matrix (\(2^{2J+2}\) entries) are nonzero. The fraction of nonzero elements is at most
	\[
	\frac{(2J+1)2^{J+1}+2}{2^{2J+2}}=(2J+1)2^{-J-1}+2^{-2J-1},
	\]
	which converges to zero. 
\end{proof}
\end{document}
